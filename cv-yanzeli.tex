% Dr. Geoff Boeing - Curriculum Vitae
% Copyright 2018 Geoff Boeing
% Email: g.boeing@northeastern.edu
% Web: https://geoffboeing.com/

\documentclass[12pt,letterpaper]{report}

\usepackage[T1]{fontenc} % output T1 font encoding (8-bit) so accented characters are a single glyph
\usepackage[utf8]{inputenc} % allow input of utf-8 encoded characters
\usepackage[strict,autostyle]{csquotes} % smart and nestable quote marks
\usepackage[english]{babel} % automatically regionalize hyphens, quote marks, etc
\usepackage{microtype} % improves text appearance with kerning, etc
\usepackage{datetime} % enable formatting of date output
\usepackage{tabto} % make nice tabbing
\usepackage{hyperref} % enable hyperlinks and pdf metadata
\usepackage{geometry} % manually set page margins
\usepackage{enumitem} % enumerate with [resume] option
\usepackage{titlesec} % allow custom section fonts
\usepackage{fancyhdr}


%\pagestyle{fancy}
%\fancyhf{}
%\lhead{U-M ID: 00828870}
%\rhead{Yanze Li, CSE Ph.D.}

% what is your name?
\newcommand{\myname}{Yanze Li}

% define a default font face and set it as the body font
\usepackage{crimson} % document's serif font
\usepackage{helvet}  % document's sans serif font

% define how far to tab for list items with left-aligned date - different font faces need different widths
\newcommand{\listtabwidth}{1.75cm}

% set name font to title the document
\newcommand{\namefont}[1]{{\normalfont\bfseries\Huge{#1}}}

% set section heading fonts and before/after spacing
\SetTracking{encoding=*}{20}
\titleformat{\section}{\sffamily\small\bfseries\lsstyle\uppercase}{}{}{}{}
\titlespacing{\section}{0pt}{24pt plus 4pt minus 2pt}{12pt plus 2pt minus 2pt}

% set subsection heading fonts and before/after spacing
\titleformat{\subsection}{\sffamily\footnotesize\bfseries}{}{}{}{}
\titlespacing{\subsection}{0pt}{12pt plus 4pt minus 2pt}{8pt plus 2pt minus 2pt}

% set page margins
\geometry{body={6.5in, 9.0in},
	left=1.0in,
	top=1.0in}

% prevent paragraph indentation
\setlength\parindent{0em}

% define space between list items
\newcommand{\listitemspace}{0.15em}

% make unordered lists without bullets and use compact spacing
\renewenvironment{itemize}
{\begin{list}{}{\setlength{\leftmargin}{0em}
			\setlength{\parskip}{0em}
			\setlength{\itemsep}{\listitemspace}
			\setlength{\parsep}{\listitemspace}}}
	{\end{list}}

% make tabbed lists so content is left-aligned next to years
\TabPositions{\listtabwidth}
\newlist{tablist}{description}{3}
\setlist[tablist]{leftmargin=\listtabwidth,
	labelindent=0em,
	topsep=0em,
	partopsep=0em,
	itemsep=\listitemspace,
	parsep=\listitemspace,
	font=\normalfont}

% print the month and year only when using \today
\newdateformat{monthyeardate}{\monthname[\THEMONTH] \THEYEAR}

% define hyperlink appearance and metadata for pdf properties
\hypersetup{
	colorlinks = true,
	urlcolor = blue,
	pdfauthor = {\myname},
	pdfkeywords = {Computer Science, Programming Language},
	pdftitle = {\myname: Curriculum Vitae},
	pdfsubject = {Curriculum Vitae},
	pdfpagemode = UseNone
}

\begin{document}
\raggedright

% display name as the document title
\namefont{\myname}

% contact info
\vspace{1em}
%	\begin{minipage}[t]{0.495\textwidth}
%		ASER, Parasol Lab \\
%		Texas A\&M University \\
%		College Station, Texas, USA
%	\end{minipage}
\begin{minipage}[c]{0.7\textwidth}
	% Location: 419B, HRBB \\
	%		Program: Computer Science \& Engineering Ph.D., Fall 2021\\
	%		U-M ID: 00828870\\
	%		Phone: +1 979 204 9448\\
	Web: \href{https://liyz.pl}{https://liyz.pl} \\
	GitHub: \href{https://github.com/funemy}{https://github.com/funemy}\\
	Personal Email: \href{mailto:liyzunique@gmail.com}{liyzunique@gmail.com} \\
	Work Email: \href{mailto:yanzeli@cs.ubc.ca}{yanzeli@cs.ubc.ca}
\end{minipage}
\vspace{-0.5em}

%-----------------------------------------------------------------------
% TODO RESEARCH INTERESTS
%-----------------------------------------------------------------------

\section*{Research Interests}

\begin{itemize}
	%		\item My area of interest is \textbf{Programming Language}.
	%		I enjoy doing research related to functional programming, program verification, type system and static analysis, etc.
	%		I believe the future programming languages will be more expressive to enable programmers to specify important properties and even synthesize implementation automatically. My goal is to applying PL techniques to facilitate programming productivity and software correctness.
	%		I used to develop program analysis tools that can scale to large codebase and various code changes with an emphasis on concurrency-related bugs (race detection, deadlock detection, etc.).
	%		Recently, I'm switching my focus to something more "formal", such as verification, type system, etc.,
	%		and I'm actively looking for PhD openings in these areas.
	%		My goal is to improve programming productivity and software reliability from a PL perspective.

	%\item My previous research focused on program analysis on concurrent programs. I developed static analysis tools that detect concurrency bugs and made them scale to large code bases. Nowadays, I'm more interested in the formal aspects of programming languages such as type theory, language semantics and design. I also enjoy applying PL techniques to guarantee the correctness of complex programs.

	%\item I'm interested in the formal aspect of programming language, such as type systems, language semantics and design, in particular, how they can facilitate tasks like verification, synthesis, resource analysis, etc. I'm also interested in program verification and static analysis, especially their applications on real complex systems.

	\item I’m currently a Ph.D. student at University of British Columbia (since Fall 2021), advised by Alexander
	      J. Summers and Ivan Beschastnikh. My research interests lie in programming languages, program
	      verification, and type theory. I’m particularly interested in how we can write simple and intuitive specifications about various program properties and prove their correctness.

\end{itemize}

%-----------------------------------------------------------------------
% TODO EDUCATION
%-----------------------------------------------------------------------


\section*{Education}

\begin{tablist}

	% \item[Ph.D.] \tab Computer Science, University of British Columbia, 2021 - Now \\
	%		Dissertation: \textit{Methods and Measures for Analyzing Complex Street Networks and Urban Form} \\
	% \textit{Advisor: Alexander J. Summers, Ivan Beschastnikh}
	\item[Ph.D.]  \tab Computer Science, University of British Columbia, 2021 - Now \\
	\textit{Advisor: Alexander J. Summers, Ivan Beschastnikh}\\
	GPA: 4.0/4.0

	\item[M.S.]  \tab Computer Science, Texas A\&M University, 2020 \\
	\textit{Thesis: Efficient and Scalable Whole Program Race Detection for Java and Android Programs}\\
	\textit{Advisor: Jeff Huang}\\
	GPA: 4.0/4.0

	\item[B.Eng.]  \tab Electrical Engineering, Huazhong University of Science and Technology, 2017 \\
	GPA: 3.67/4.0 \hspace{0.5cm} Major GPA: 3.81/4.0

\end{tablist}


%-----------------------------------------------------------------------
% TODO PUBLICATION
%-----------------------------------------------------------------------


\section*{Publications}
\renewcommand{\listtabwidth}{3cm}
\begin{tablist}
	\item[ICSE'22] \tab \href{https://liyz.pl/index/icse22-pus.pdf}{\textit{\enquote{PUS: A Fast and Highly Efficient Solver for Inclusion-based Pointer Analysis}}}\\
	Peiming Liu, \textbf{Yanze Li}, Bradley Swain, Jeff Huang \\
	\textit{International Conference on Software Engineering (ICSE'22). 2022.} \\
	\textit{\textbf{ACM SIGSOFT Distinguished Paper Award}}
	\item[Correctness'21] \tab \href{https://liyz.pl/index/correctness21-openrace.pdf}{\textit{\enquote{OpenRace: An Open Source Framework for Statically Detecting Data Races}}}\\
	Bradley Swain, Jeff Huang, Bozhen Liu, Peiming Liu, \textbf{Yanze Li}, Addison Crump, Rohan Khera \\
	\textit{2021 IEEE/ACM 5th International Workshop on Software Correctness for HPC Applications (Correctness). IEEE, 2021.}
	\item[PLDI'21] \tab \href{https://liyz.pl/index/pldi21-origin.pdf}{\textit{\enquote{When Threads Meet Events: Efficient and Precise Static Race Detection with Origins}}}\\
	Bozhen Liu, Peiming Liu, \textbf{Yanze Li}, Chia-Che Tsai, Dilma Da Silva, Jeff Huang \\
	\textit{42nd ACM SIGPLAN International Conference on Programming Language Design and Implementation. 2021.}
	\item[SC'20] \tab \href{https://liyz.pl/index/sc20-ompracer.pdf}{\textit{\enquote{OMPRacer: A Scalable and Precise Static Race Detector for OpenMP Programs}}}\\
	Bradley Swain, \textbf{Yanze Li}, Peiming Liu, Ignacio Laguna, Giorgis Georgakoudis, Jeff Huang\\
	\textit{International Conference for High Performance Computing, Networking, Storage and Analysis. IEEE, 2020.}
	\item[ICSE'19] \tab (Demo Track) \href{https://liyz.pl/index/icse2019-demo.pdf}{\textit{\enquote{SWORD: A Scalable Whole Program Race Detector for Java}}}\\
	\textbf{Yanze Li}, Bozhen Liu, Jeff Huang\\
	\textit{2019 IEEE/ACM 41st International Conference on Software Engineering: Companion Proceedings (ICSE-Companion). IEEE, 2019.}
\end{tablist}
\renewcommand{\listtabwidth}{1.75cm}

%-----------------------------------------------------------------------
% TODO Professional Experiences
%-----------------------------------------------------------------------

\section*{Research Experience}
\begin{tablist}[style=multiline, leftmargin=*]
	\item[2021.9- Now]
	\tab \textbf{Research Assistant, University of British Columbia, Canada}\\
	\tab Working on automated verification of liveness guarantees in async runtime systems. \\
	\tab Currently I'm focusing on formally verifying certain liveness properties in different Rust \\
	\tab async runtime implementations and automating such verification using static analysis.\\
	\item[2020.8- 2021.6]
	\tab \textbf{Research Intern (Remote), Utrecht University, Netherland}\\
	\tab Worked with Dr. Jurriaan Hage on the LLVM backend and FFI of a Haskell compiler \\
	\tab called Helium.
	\item[2018.6- 2020.6]
	\tab \textbf{Research Assistant, Texas A\&M University, USA}\\
	\tab Worked on static analysis for concurrent programs. Developed tools that scale to million\\
	\tab lines of Java/C++/Android code and efficiently detect potential data races and deadlocks.
\end{tablist}

% \pagebreak

\section*{Work Experience}

\begin{tablist}[style=multiline, leftmargin=*]
	\item[2019.7- 2021.5]
	\tab \textbf{Software Engineer, Coderrect Inc., USA}\\
	\tab (As intern during 2019.7-8, 2020.2-7, as full-time employee during 2020.10-2021.5) \\
	\tab Worked as the main developer of an LLVM-based program analysis tool for detecting\\
	\tab concurrency bugs and anti-patterns in C/C++/Fortran/CUDA code. I designed a highly\\
	\tab efficient static happens-before graph, lock tracking algorithm and race detection algorithm\\
	\tab which enable the tool to analyze million lines of code in minutes accurately.
	\item[2015.11- 2017.4]
	\tab \textbf{Software Engineer, Nightingale Technology, China}\\
	\tab Worked on a second-hand commodities trading platform for college students and an\\
	\tab integrated web application for editing and publishing news articles as well as managing\\
	\tab and visualizing their statistics.
\end{tablist}

\section*{Teaching Experience}
% \renewcommand{\listtabwidth}{2.5cm}
\begin{tablist}
	\item[2024S] \tab CPSC 410: Advanced Software Engineering, Teaching Assistant
	\item[2023F] \tab CPSC 539S: Program Verifiers and Program Verification, Teaching Assistant
	\item[2022F] \tab CPSC 410: Advanced Software Engineering, Teaching Assistant
	\item[2022S] \tab CPSC 416: Distributed Systems, Teaching Assistant
	\item[2021F] \tab CPSC 410: Advanced Software Engineering, Teaching Assistant
\end{tablist}
% \renewcommand{\listtabwidth}{1.75cm}

\vspace{-0.5em}

\renewcommand{\listtabwidth}{2.8cm}
\section*{Projects}

\begin{tablist}
	\item[\textbf{TraceChecker}]
	\tab A DSL for distributed system trace verification. Used as a teaching tool to grade students' assignments based on formal specifications.
	\href{https://github.com/DistributedClocks/TraceChecker}{[GitHub]}
	\item[\textbf{LTLSpec}]
	\tab A proof-of-concept Haskell framework for modelling, specifying, and verifying distributed system traces in linear temporal logic.
	\href{https://github.com/ejconlon/ltlspec}{[GitHub]}
	\item[\textbf{Helium}]
	\tab A compiler for a subset of Haskell that aims at delivering high quality type error messages particularly for beginner programmers. It also includes facilities for specializing type error diagnosis for embedded domain specific languages. \href{https://github.com/Helium4Haskell/helium}{[GitHub]}
	\item[\textbf{Coderrect}]
	\tab An LLVM-based static analyzer, specialize in detecting concurrency related bugs and anti-patterns, found several previously unkown bugs in Linux kernel, Redis, memcached, and GraphBLAS. \href{https://coderrect.com/}{[Website]} \href{https://github.com/jtao/OpenRace}{[GitHub]}
	\item[\textbf{OMPRacer}]
	\tab An LLVM-based race detector for OpenMP programs,
	using inter-procedure value-flow analysis to reason about array accesses.
	Found several previously unknown bugs in ECP proxy applications and a major simulator for COVID-19. \href{https://github.com/parasol-aser/OMPRacer}{[GitHub]}
	\item[\textbf{Crappie}]
	\tab An incremental race detection engine that scales to distributed systems and Android apps and has been implemented as an Intellij IDEA plugin.
	\item[\textbf{SWORD}]
	\tab A whole program race detector for Java (source code/bytecode) and has been implemented as an Eclipse plugin. \href{https://github.com/funemy/SWORD}{[GitHub]}
\end{tablist}
\renewcommand{\listtabwidth}{1.75cm}

%-----------------------------------------------------------------------
% TODO AWARD
%-----------------------------------------------------------------------
\vspace{-0.5em}
\section*{Honor and Awards}

\begin{tablist}
	\item[2022] \tab ACM SIGSOFT Distinguished Paper Award
	\item[2022] \tab OPLSS Fellowship Grant
	\item[2019] \tab ACM SIGSOFT CAPS Award
	\item[2017] \tab Excellent Graduated Student at HUST
	\item[2015] \tab Scientific Research Innovation Scholarship
	\item[2014] \tab 3\textsuperscript{rd} place, China University Cloud Computing Innovation Competition
\end{tablist}


%-----------------------------------------------------------------------
% TODO SEVERICE
%-----------------------------------------------------------------------
\vspace{-0.5em}
\section*{Service}
\begin{tablist}[style=multiline, leftmargin=*]
	\item[2020.8- 2022.11] \tab SIGPLAN Long-Term Mentoring Program (SIGPLAN-M), Operations Team
\end{tablist}

\section*{Sub-Reviewer}
\begin{tablist}
	\item[2023] \tab ICSE
	\item[2022] \tab ASE
	\item[2020] \tab OOPSLA
	\item[2019] \tab PLDI, ICSE, FSE, OOPSLA
	\item[2018] \tab TSE
\end{tablist}


%	\subsection*{Service to Field}
%
%	\begin{itemize}
%
%		\item Review and Appraisal Committee, Association of Collegiate Schools of Planning, 2018--
%
%		\item Scientific Committee, Symposium on Simulation for Architecture and Urban Design, 2018--
%
%		\item Book review editor, \textit{Berkeley Planning Journal}, 2013--14
%
%	\end{itemize}
%
%	\subsection*{Service to Department}
%
%	\begin{itemize}
%
%		\item Standing Committee on Digital Proficiencies and Quantitative Methods, NU CSSH, 2018--
%
%		\item Ph.D. program faculty representative, UC Berkeley DCRP, 2015--16
%
%		\item Ph.D. program admissions committee, UC Berkeley DCRP, 2015--16
%
%	\end{itemize}



%	\section*{Professional Affiliations}
%
%	\begin{itemize}
%
%		\item American Association of Geographers
%
%		\item American Planning Association
%
%		\item Association of Collegiate Schools of Planning
%
%		\item Association for Computing Machinery
%
%		\item Complex Systems Society
%
%		\item Regional Studies Association
%
%		\item New York Academy of Sciences
%
%		\item Project Management Institute
%
%		\item Python Software Foundation
%
%		\item Urban Affairs Association
%
%	\end{itemize}



%	\section*{Credentials}
%
%	\begin{itemize}
%
%		\item U.S. Department of Defense secret clearance
%
%		\item U.S. Department of Homeland Security public trust
%
%		\item Project Management Professional (PMP)
%
%	\end{itemize}
%
%
%
%	\section*{Consulting Engagements}
%
%	\begin{tablist}
%
%		\item[2017--19] \tab The Public Good Projects
%
%		\item[2017--18] \tab Calthorpe Analytics
%
%		\item[2016--18] \tab UrbanSim Inc.
%
%		\item[2013--18] \tab Avalon Health Economics
%
%		\item[2013]     \tab Raimi \& Associates
%
%		\item[2009--13] \tab Accenture
%
%	\end{tablist}

%-----------------------------------------------------------------------
% TODO EXPERIENCE
%-----------------------------------------------------------------------

%	\section*{Research Experience}
%
%	\begin{tablist}
%
%		\item[2017] \tab Accenture, Consultant/Project Manager. London, England; New York, New York; San Diego, California.
%
%		\item[2016] \tab Permission Data, Front-End Systems Product Manager. New York, New York.
%
%	\end{tablist}



%	\section*{Selected Media Coverage}
%
%	Complete listing available at https://geoffboeing.com/press/ \bigskip
%
%	\begin{tablist}
%
%		\item 2018 \tab \textit{CityLab}. \enquote{Visualizing the Hidden Logic of Cities} Jul 26.
%
%		\item 2018 \tab \textit{New Statesman CityMetric}. \enquote{Do British Cities Have Grid Systems? We Used Science to Find Out.} Jul 23.
%
%		\item 2018 \tab \textit{99 Percent Invisible}. \enquote{On the Grid: Visualizing Street Network Orientations Across 50 Global Cities.} Jul 20.
%
%		\item 2018 \tab \textit{Fast Company}. \enquote{How Crazy Is Your City's Plan?} Jul 16.
%
%		\item 2018 \tab \textit{The Boston Globe}. \enquote{Boston's Streets Do Go in All Sorts of Directions.} Jul 12.
%
%		\item 2018 \tab \textit{Slate}. \enquote{Elegant Graphs Reduce 25 American Cities to Their Design Essence.} Jul 11.
%
%		\item 2018 \tab \textit{Chicago Magazine}. \enquote{What Craigslist Can Tell Us About Rents in Chicago.} Jan 24.
%
%		\item 2017 \tab \textit{The San Francisco Chronicle}. \enquote{Stunning, Simple Maps Show San Francisco versus Other Global Cities.} Jun 19.
%
%		\item 2017 \tab \textit{The Daily Mail}. \enquote{Square Mile Maps Reveal How Different the World's Cities Really Are.} Jun 9.
%
%		\item 2017 \tab \textit{Forbes}. \enquote{Understanding Our Cities, Thanks to Beautiful Maps.} Feb 7.
%
%		\item 2017 \tab \textit{Fast Company}. \enquote{Turn Your Local Streets into a Map That Reveals the Character of Your Neighborhood.} Feb 6.
%
%		\item 2017 \tab \textit{Domus Magazine}. \enquote{Do-It-Yourself City Mapping.} Jan 23.
%
%		\item 2017 \tab \textit{CityLab}. \enquote{A Digital Window into Your City's Urban Form.} Jan 17.
%
%		\item 2017 \tab \textit{Discovery News}. \enquote{Compare City Street Grids One Square Mile at a Time.} Jan 9.
%
%		\item 2016 \tab \textit{The Washington Post}. \enquote{What More Than 1 Million Craigslist Rental Listings Tell Us about the Housing Market.} Sep 1.
%
%		\item 2016 \tab \textit{Fast Company}. \enquote{11 Million Craigslist Ads Show Which Cities Have the Highest Rents.} Sep 1.
%
%		\item 2016 \tab \textit{NextCity}. \enquote{What 11 Million Craigslist Posts Show About Affordable Housing.} Aug 26.
%
%	\end{tablist}
%

%-----------------------------------------------------------------------
% TODO SKILL
%-----------------------------------------------------------------------

% \section*{Skills}

% \subsection*{Research}

% \begin{itemize}

% 	\item Static Analysis, Concurrency

% \end{itemize}

% \subsection*{Languages}

% \begin{itemize}

% 	\item C++, Agda, OCaml, Java, JavaScript, Python, Ruby

% \end{itemize}

% \subsection*{Misc}
% \begin{itemize}
% 	\item LLVM, WALA, Soot, Doop
% \end{itemize}





\end{document}
